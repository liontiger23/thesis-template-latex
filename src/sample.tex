% Options for packages loaded elsewhere
\PassOptionsToPackage{unicode}{hyperref}
\PassOptionsToPackage{hyphens}{url}
%
\documentclass[
  14pt,
  a4paper,
]{extarticle}
\usepackage{amsmath,amssymb}
\usepackage{iftex}
\ifPDFTeX
  \usepackage[T1]{fontenc}
  \usepackage[utf8]{inputenc}
  \usepackage{textcomp} % provide euro and other symbols
\else % if luatex or xetex
  \usepackage{unicode-math} % this also loads fontspec
  \defaultfontfeatures{Scale=MatchLowercase}
  \defaultfontfeatures[\rmfamily]{Ligatures=TeX,Scale=1}
\fi
\usepackage{lmodern}
\ifPDFTeX\else
  % xetex/luatex font selection
  \setmainfont[]{Times New Roman}
  \setmonofont[]{DejaVu Sans Mono}
  \setmathfont[]{XITS Math}
\fi
% Use upquote if available, for straight quotes in verbatim environments
\IfFileExists{upquote.sty}{\usepackage{upquote}}{}
\IfFileExists{microtype.sty}{% use microtype if available
  \usepackage[]{microtype}
  \UseMicrotypeSet[protrusion]{basicmath} % disable protrusion for tt fonts
}{}
\usepackage{xcolor}
\usepackage[left=3cm,right=2cm,top=3cm,bottom=2cm]{geometry}
\usepackage{listings}
\newcommand{\passthrough}[1]{#1}
\lstset{defaultdialect=[5.3]Lua}
\lstset{defaultdialect=[x86masm]Assembler}
\usepackage{longtable,booktabs,array}
\usepackage{calc} % for calculating minipage widths
% Correct order of tables after \paragraph or \subparagraph
\usepackage{etoolbox}
\makeatletter
\patchcmd\longtable{\par}{\if@noskipsec\mbox{}\fi\par}{}{}
\makeatother
% Allow footnotes in longtable head/foot
\IfFileExists{footnotehyper.sty}{\usepackage{footnotehyper}}{\usepackage{footnote}}
\makesavenoteenv{longtable}
\usepackage{graphicx}
\makeatletter
\def\maxwidth{\ifdim\Gin@nat@width>\linewidth\linewidth\else\Gin@nat@width\fi}
\def\maxheight{\ifdim\Gin@nat@height>\textheight\textheight\else\Gin@nat@height\fi}
\makeatother
% Scale images if necessary, so that they will not overflow the page
% margins by default, and it is still possible to overwrite the defaults
% using explicit options in \includegraphics[width, height, ...]{}
\setkeys{Gin}{width=\maxwidth,height=\maxheight,keepaspectratio}
% Set default figure placement to htbp
\makeatletter
\def\fps@figure{htbp}
\makeatother
\setlength{\emergencystretch}{3em} % prevent overfull lines
\providecommand{\tightlist}{%
  \setlength{\itemsep}{0pt}\setlength{\parskip}{0pt}}
\setcounter{secnumdepth}{5}
\usepackage[style=common/gost/gost-r-7-0-5-2008-numeric]{citation-style-language}
\addbibresource{common/citations.bib}
\ifLuaTeX
\usepackage[bidi=basic]{babel}
\else
\usepackage[bidi=default]{babel}
\fi
\babelprovide[main,import]{russian}
\ifPDFTeX
\else
\babelfont{rm}[]{Times New Roman}
\fi
% get rid of language-specific shorthands (see #6817):
\let\LanguageShortHands\languageshorthands
\def\languageshorthands#1{}

%%%%%%%%%%%%%%%%%%%
%% Formatting for GOST
%%%%%%%%%%%%%%%%%%%

%% Notes in pdf
\usepackage{pdfcomment}

%% Support for \mathbb and \mathcal
\DeclareMathAlphabet{\mathcal}{OMS}{cmsy}{m}{n}
\let\mathbb\relax % remove the definition by unicode-math
\DeclareMathAlphabet{\mathbb}{U}{msb}{m}{n}

%% Page numbering
\usepackage{fancyhdr}
\pagestyle{fancy}
\fancyhf{}
\renewcommand{\headrulewidth}{0pt}
%\fancyhead[C]{\thepage}
\cfoot{\thepage}

%% Enable catenation of pdfs (e.g. prpending title.pdf to rest of the text)
\usepackage{pdfpages}

%% Start page numbering from 2, due to title being the first one
%% Uncomment if title pdf is not prepended in include-before clause in yaml
%\setcounter{page}{2}

%% Double spaces
\usepackage{setspace}
\doublespacing
%\onehalfspacing

%% Indents
\usepackage{indentfirst}

%% Center headings
\usepackage[center]{titlesec}
\titlelabel{\thetitle.\quad}

%% Magic option for dots in labels
\usepackage[dotinlabels]{titletoc}

\usepackage[center,font=singlespacing]{caption}

%% Proper names
\renewcommand{\lstlistingname}{Листинг}

%% Within-section numbering
\numberwithin{table}{section}
\numberwithin{figure}{section}
\numberwithin{equation}{section}
%\usepackage{chngcntr}
%\renewcommand{\counterwithin}{\@ifstar{\@csinstar}{\@csin}}
\AtBeginDocument{\numberwithin{lstlisting}{section}}

%% Cyrillic numbering of subfigures
\usepackage{subfig}
\makeatletter
\def\cyralph#1{\expandafter\@cyralph\csname c@#1\endcsname}
\def\@cyralph#1{\ifcase #1\or а\or б\or в\or г\or д\or е\or ж\or з\or и\or к\or л\or м\or н\or о\or п\or р\or с\or т\or у\or ф\or х\or ц\or ч\or ш\or щ\or э\or ю\or я\else \@ctrerr \fi}
\renewcommand{\thesubfigure}{\cyralph{subfigure}}
\makeatother

%% Theorem and other stuff
\newcounter{stmt}
\numberwithin{stmt}{section}

%% Hyphenation rules
\AtBeginDocument{\selectlanguage{russian}\hyphenation{также су-пер-класс су-пер-ин-тер-фейс}}

\usepackage{lscape}

%% Define \ceil*{} -- https://tex.stackexchange.com/questions/42271/floor-and-ceiling-functions/42274
\usepackage{mathtools}
\DeclarePairedDelimiter\ceil{\lceil}{\rceil}

%%%%
%% Threorm, algorithm and other stuff
%%%%

\usepackage{amsthm}

\newtheorem{algorithmbase}{Алгоритм}[section]
\newtheorem{definition}{Определение}[section]
\newtheorem{theorem}{Теорема}[section]
\newtheorem{lemma}[theorem]{Лемма}
\newtheorem{statement}[theorem]{Утверждение}
\newtheorem{corollary}[theorem]{Следствие}

%% Fix spacing between nested items and outer ones
\usepackage{enumitem}
\setlist[enumerate]{topsep=\parskip,itemsep=\parskip}
\setlist[itemize]{topsep=\parskip,itemsep=\parskip}

%\AtBeginDocument{
%    \abovedisplayskip=2\parskip
%    \belowdisplayskip=2\parskip
%    \abovedisplayshortskip=2\parskip
%    \belowdisplayshortskip=2\parskip
%}

\newenvironment{algorithm}[1]{
    \begin{algorithmbase}\label{#1}
    %% nested enumerate -- https://stackoverflow.com/a/24650181
    \renewcommand{\labelenumi}{\arabic{enumi}.} 
    \renewcommand{\labelenumii}{\arabic{enumi}.\arabic{enumii}}
    \renewcommand{\labelenumiii}{\arabic{enumi}.\arabic{enumii}.\arabic{enumii}}
} {
    \qed
    \end{algorithmbase}
}

%%%%%%%%%%%

%%%%%%%%%%%%%%%%%%%
%% Listings
%%%%%%%%%%%%%%%%%%%

%\usepackage{xcolor}   % for \textcolor

%% Fix for "Missing number, treated as zero spaces"
%% when using consecutive spaces
\newsavebox\grayarrow
\sbox\grayarrow{\raisebox{0ex}[0ex][0ex]{\ensuremath{\hookrightarrow\space}}}

\lstdefinestyle{default}{
    basicstyle   = {\linespread{1}\ttfamily},
    %% Uncomment to move caption below listing
    %captionpos   = b,
    breaklines   = true,
    numbersep    = 5pt,
    escapeinside = {\$\$},
    columns      = space-flexible,
    postbreak    = \usebox\grayarrow,
    showstringspaces = false,
    texcl        = true,
    %% Examples of custom keyword highlighting
    %emph         = [1]{ordered,by,break,dom,call},
    %emphstyle    = [1]{\bfseries},
    %emph         = [2]{super,final,protected,private,package},
    %emphstyle    = [2]{},
    %emph         = [3]{find,resolve},
    %emphstyle    = [3]{\itshape},
}

\lstset{style=default}

%%%%%%%
%% Allow skipping line numbers -- https://tex.stackexchange.com/a/215752
\let\origthelstnumber\thelstnumber
\makeatletter
\newcommand*\Suppressnumber{%
  \lst@AddToHook{OnNewLine}{%
    \let\thelstnumber\relax%
     \advance\c@lstnumber-\@ne\relax%
    }%
}

\newcommand*\Reactivatenumber[1]{%
  \lst@AddToHook{OnNewLine}{%
   \let\thelstnumber\origthelstnumber%
   %\setcounter{lstnumber}{\numexpr#1-1\relax}%
   \advance\c@lstnumber\@ne\relax%
  }%
}

\makeatother
%%%%%%%


%%%%%%%%%%%%%%%%%%%
%% Other
%%%%%%%%%%%%%%%%%%%

%% Workaround for pandoc-crossref issue - https://github.com/lierdakil/pandoc-crossref/issues/326
\makeatletter
\@ifpackageloaded{subfig}{}{\usepackage{subfig}}
\@ifpackageloaded{caption}{}{\usepackage{caption}}
\captionsetup[subfloat]{margin=0.5em}
\makeatother

\usepackage{mdframed}

%% Provides BVerbatim environment which boxes verbatim
%% so it can be centered -- https://tex.stackexchange.com/a/122197
\usepackage{fancyvrb}

%% Set numbered footnotes in minipages
\renewcommand{\thempfootnote}{\arabic{mpfootnote}}

%% Do not number tables
%\captionsetup[table]{labelformat=empty}

%% Enable strikethrough syntax \st of newer pandoc
\usepackage{soul}

\usepackage{csquotes}

%% Normal subscripts in listings using \textsubscript{i} -- https://tex.stackexchange.com/questions/63845/boldface-and-subscripts-in-verbatim-mode    
\usepackage{fixltx2e}

%% Hack for using latex envs inside markdown -- https://github.com/jgm/pandoc/issues/3145#issuecomment-302787889
\newcommand{\hideFromPandoc}[1]{#1}
\hideFromPandoc{
  \let\Begin\begin
  \let\End\end
}

%% TODO command
\newcommand\todo[1]{\textcolor{red}{#1}}

%%%%%%%%%%%%%%%%%%%
%% Tikz stuff
%%%%%%%%%%%%%%%%%%%

\usepackage{tikz}
\usetikzlibrary{tikzmark,shapes,matrix,calc,positioning,fit,graphs,arrows.meta,backgrounds,decorations.pathreplacing}

\usetikzmarklibrary{listings}

\tikzset{>=latex}

%% Align text vertically with subscript
%\tikzset{text depth=.25ex}
%% Align text vertically without subscript
\tikzset{text depth=0}


%% Scope prefix support -- https://tex.stackexchange.com/a/128079
\makeatletter
\tikzset{%
  prefix/.code={%
    \tikzset{%
      name/.code={\edef\tikz@fig@name{#1 ##1}}
    }%
  }%
}
\makeatother

%% Tikz figure
\newenvironment{tikzfigure}[3]{
    \def\figlabel{#1}
    \def\figcaption{#2}

    \begin{figure}
    \centering
    \begin{tikzpicture}[#3]
} {
    \end{tikzpicture}
    \caption{\figcaption}
    \label{\figlabel}
    \end{figure}
}
\newcommand{\tikzsubfloat}[3]{
    \subfloat[#2]{
    \begin{tikzpicture}[baseline=(current bounding box.north)]
    #3
    \end{tikzpicture}
    \label{#1}
    }
}



%%%%%%%%%%%%%%%%%%%
%% pgfplots stuff
%%%%%%%%%%%%%%%%%%%

\usepackage{pgfplots}
\usepackage{pgfplotstable}

\definecolor{googleblue}{HTML}{4285F4}
\definecolor{googlered}{HTML}{EA4335}
\definecolor{googleyellow}{HTML}{FBBC04}
\definecolor{googlegreen}{HTML}{34A753}
\definecolor{googleorange}{HTML}{FE6D00}

\ifLuaTeX
  \usepackage{selnolig}  % disable illegal ligatures
\fi
\usepackage{bookmark}
\IfFileExists{xurl.sty}{\usepackage{xurl}}{} % add URL line breaks if available
\urlstyle{same}
\hypersetup{
  pdflang={ru-RU},
  hidelinks,
  pdfcreator={LaTeX via pandoc}}

\author{}
\date{}


\begin{document}

\includepdf[pages={1}]{src/title.pdf}

\renewcommand*\contentsname{СОДЕРЖАНИЕ}
{
\setcounter{tocdepth}{3}
\tableofcontents
}
\pagebreak

\section*{ВВЕДЕНИЕ}\label{sec:introduction}
\addcontentsline{toc}{section}{ВВЕДЕНИЕ}

Данная работа является примером оформления текста ВКР. В качестве
примера ссылки на литературу, можно упомянуть язык Java~\cite{gosling2000java},
который исполняется на виртуальной машине JVM~\cite{lindholm2014java}.

Завершение введения отдельным параграфом со сноской\footnote{Любая
  дополнительная информация может вынесена в сноску, включая
  форматирование \emph{текста} и формул (\(\sum_{i}^{n}{i}\)).}.

\clearpage
\pagebreak

\section{ПОСТАНОВКА ЗАДАЧИ}\label{sec:overview}

Постановка задачи и обзор существующих работ.

\subsection{Существующие работы}\label{sec:related-work}

Эта секция исключительно для примеров вложенных списков.

\begin{enumerate}
\def\labelenumi{\arabic{enumi}.}
\tightlist
\item
  Первый элемент
\item
  Второй элемент

  \begin{enumerate}
  \def\labelenumii{\arabic{enumii}.}
  \tightlist
  \item
    Первый вложенный элемент
  \item
    Второй вложенный элемент
  \end{enumerate}
\item
  Третий элемент
\end{enumerate}

И перечисление без порядка:

\begin{itemize}
\tightlist
\item
  Какой-то элемент
\item
  Следующий элемент

  \begin{itemize}
  \tightlist
  \item
    Новый вложенный элемент
  \item
    Еще вложенный элемент
  \end{itemize}
\item
  Последний элемент
\end{itemize}

А также вперемешку:

\begin{itemize}
\tightlist
\item
  Какой-то элемент
\item
  Следующий элемент

  \begin{enumerate}
  \def\labelenumi{\arabic{enumi}.}
  \tightlist
  \item
    Первый вложенный элемент
  \item
    Второй вложенный элемент
  \end{enumerate}
\item
  Последний элемент
\end{itemize}

\subsection{Теорема Пифагора}\label{sec:pythagoras}

Основная формулировка содержит алгебраические действия --- в
прямоугольном треугольнике, длины катетов которого равны \(a\) и \(b\),
а длина гипотенузы --- \(c\), выполнено соотношение:

\[
a^2 + b^2 = c^2.
\]

Для того чтобы ссылаться на формулы, их можно нумеровать следующим
образом:

\begin{equation}\label{eq:pythagoras}{
a^2 + b^2 = c^2
}\end{equation}

Теперь можно сослаться на формулу~\eqref{eq:pythagoras} где угодно в
тексте.

\subsection{Пример листинга}\label{sec:lst-example}

Ниже в листинге~\ref{lst:python-fact} представлен пример вычисления
факториала на языке Python.

\begin{lstlisting}[language=Python, caption={Вычисление факториала числа n}, label=lst:python-fact]
def fact(n):
  if (n==1 or n==0):
    return 1
  else:
    return n * fact(n - 1)
\end{lstlisting}

\subsection{Пример рисунка}\label{sec:fig-example}

Далее на рис.~\ref{fig:png-svg-compare}~и~\ref{fig:tikz-example}
представлены примеры вставки изображений в работу.

\begin{figure}\centering%

\subfloat[]{\includegraphics[width=0.4\textwidth,height=\textheight]{images/sample/Markdown-mark.svg.png}\label{fig:markdown-mark-png}}
\subfloat[]{\includegraphics{images/sample/Markdown-mark.pdf}\label{fig:markdown-mark-svg}}

\caption{Пример рисунка в формате png \ref{fig:markdown-mark-png} и в
формате svg после конвертации в pdf
\ref{fig:markdown-mark-svg}}\label{fig:png-svg-compare}

\end{figure}

Всегда лучше выбирать рисунки в векторном формате (.svg, .pdf и.т.п),
либо рисовать прямо в \LaTeX\ с помощью Ti\emph{k}Z, как показано на
рис.~\ref{fig:tikz-example}.

\begin{tikzfigure}{fig:tikz-example}{Таблица виртуальных методов для класса C}
  {every node/.style={text=black}}

  \begin{scope}[blend group = soft light]
    \fill[red!30!white]   ( 90:1.2) circle (2);
    \fill[green!30!white] (210:1.2) circle (2);
    \fill[blue!30!white]  (330:1.2) circle (2);
  \end{scope}
  \node at ( 90:2)    {Typography};
  \node at ( 210:2)   {Design};
  \node at ( 330:2)   {Coding};
  \node [font=\Large] {\LaTeX};

\end{tikzfigure}

\clearpage
\pagebreak

\section{РАЗРАБОТАННЫЙ ПОДХОД}\label{sec:solution}

\subsection{Определения}\label{sec:definitions}

\begin{definition}\label{def:even} Четным числом называется целое число, которое представимо
в виде произведения числа 2 и другого целого числа.
\end{definition}

\begin{definition}\label{def:odd} Нечетным числом называется целое
число, которое \textbf{не} представимо в виде произведения числа 2 и
другого целого числа.
\end{definition}

\subsection{Теоремы}\label{sec:theorems}

\begin{theorem}\label{thm:even} Сумма двух четных целых всегда четная.
\end{theorem}

\begin{proof} Пусть даны два целых числа $x, y \in \mathbb{Z}$. По определению\ \ref{def:even},
существуют такие целые числа $a, b \in \mathbb{Z}$, что $x$ и $y$ могут быть записаны следующим образом:
$$
\begin{aligned}
x &= 2a \\
y &= 2b
\end{aligned}
$$

Тогда их сумма также удовлетворяет определению\ \ref{def:even},
а значит является четным числом:
$$
x + y = 2a + 2b = 2(a + b).
$$
\end{proof}

\begin{corollary}\label{cor:even} Сумма $n$ четных целых всегда четная.
\end{corollary}

\subsection{Алгоритмы}\label{sec:algorithms}

Далее приведен алгорим~\ref{alg:gcd} для вычисления наибольшего общего
делителя.

\begin{algorithm}{alg:gcd} Поиск наибольшего общего делителя (НОД) двух
неотрицательных целых чисел \(a\) и \(b\), где \(a > b\).

\begin{enumerate}
\tightlist
\item
  Инициализировать переменные \(a\) и \(b\) аргументами;
\item
  Пока \(b \neq 0\):

  \begin{enumerate}
  \tightlist
  \item
    Вычислить остаток \(r\) от деления \(a\) на \(b\):
    \(r \gets a \mod b\);
  \item
    Обновить переменные \(a \gets b\), \(b \gets r\);
  \end{enumerate}
\item
  Вернуть значение \(a\) в качестве результата.
\end{enumerate}

\end{algorithm}

\clearpage
\pagebreak

\section{РЕЗУЛЬТАТЫ}\label{sec:results}

Тут приводится некоторое описание проведенных замеров, окружение, в
котором они проводились и представляются итоговые результаты\footnote{В
  тексте работы лучше приводить не только отрендеренные графики, но и
  таблицы с финальными данными, а также доверительные интервалы для
  каждой метрики.}.

\begin{longtable}[]{@{}lrrr@{}}
\caption{Замеры производительности бенчмарка
\passthrough{\lstinline!Bench!}}\tabularnewline
\toprule\noalign{}
Версия & Время (сек) & Кол-во запросов в сек. & Точность (\%) \\
\midrule\noalign{}
\endfirsthead
\toprule\noalign{}
Версия & Время (сек) & Кол-во запросов в сек. & Точность (\%) \\
\midrule\noalign{}
\endhead
\bottomrule\noalign{}
\endlastfoot
Base & 12.34 & 567 & 0.89 \\
Opt & 10.12 & 600 & 0.95 \\
Opt+ & \textbf{9.87} & \textbf{610} & \textbf{0.98} \\
Deopt & 15.67 & 520 & 0.75 \\
\end{longtable}

Также приводится анализ полученных результатов, в котором полезно
подчеркнуть не только замеры на которых произошли улучшения или
ухудшения, но и почему они там произошли, и как можно было бы этого
избежать.

\clearpage
\pagebreak

\section*{ЗАКЛЮЧЕНИЕ}\label{sec:conclusion}
\addcontentsline{toc}{section}{ЗАКЛЮЧЕНИЕ}

Основные результаты работы:

\begin{itemize}
\tightlist
\item
  Первый результат;
\item
  Второй результат;
\item
  Третий результат;
\item
  Последний результат.
\end{itemize}

Направление дальнейших работ:

\begin{itemize}
\tightlist
\item
  Улучшить алгоритм;
\item
  Провести анализ данных.
\end{itemize}

\clearpage
\pagebreak

\section*{СПИСОК ИСПОЛЬЗОВАННЫХ ИСТОЧНИКОВ}\label{sec:bib}
\addcontentsline{toc}{section}{СПИСОК ИСПОЛЬЗОВАННЫХ ИСТОЧНИКОВ}

\printbibliography[heading=none]

\renewcommand{\thesection}{П}
\setcounter{subsection}{0}
\setcounter{figure}{0}
\setcounter{lstlisting}{0}
\pagebreak

\section*{ПРИЛОЖЕНИЕ}\label{sec:appendix}
\addcontentsline{toc}{section}{ПРИЛОЖЕНИЕ}

\subsection{Первая глава приложения}\label{sec:appendix-1}

В приложение обычно выносятся длинные листинги и таблицы.

\end{document}
